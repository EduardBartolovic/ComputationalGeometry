\documentclass[conference]{IEEEtran}
%\IEEEoverridecommandlockouts
% The preceding line is only needed to identify funding in the first footnote. If that is unneeded, please comment it out.
\usepackage{cite}
\usepackage[ngerman]{babel}
\usepackage[utf8]{inputenc}
\usepackage{amsmath,amssymb,amsfonts}
\usepackage{algorithmic}
\usepackage{graphicx}
\usepackage{textcomp}
\usepackage{xcolor}
\usepackage{listings}


\definecolor{pblue}{rgb}{0.13,0.13,1}
\definecolor{pgreen}{rgb}{0,0.5,0}
\definecolor{pred}{rgb}{0.9,0,0}
\definecolor{pgrey}{rgb}{0.46,0.45,0.48}
\lstset{language=Java,
	showspaces=false,
	showtabs=false,
	breaklines=true,
	tabsize=2,
	showstringspaces=false,
	breakatwhitespace=true,
	commentstyle=\color{pgreen},
	keywordstyle=\color{pblue},
	stringstyle=\color{pred},
	basicstyle=\ttfamily
}


\usepackage{url}
\def\BibTeX{{\rm B\kern-.05em{\sc i\kern-.025em b}\kern-.08em
		T\kern-.1667em\lower.7ex\hbox{E}\kern-.125emX}}
\begin{document}
	
	\title{Computational Geometry - Abgabe 3}
	
	\author{\IEEEauthorblockN{1\textsuperscript{st} Bartolovic Eduard}
		\IEEEauthorblockA{\textit{Hochschule München} \\
			München, Deutschland \\
			eduard.bartolovic0@hm.edu}
	}
	
	\maketitle
	
	%\begin{abstract}	
	%\end{abstract}


	\section{Schnittpunkt zwischen 2 Strecken }
	In der ersten Abgabe wurde ein Verfahren entwickelt, welches prüft, ob sich 2 Strecken schneiden. Für den Bentley-Ottmann Algorithmus reicht dies aber nicht. Für diesen muss auch der genaue Schnittpunkt bekannt sein.
	Dafür wurde ein neuer Algorithmus entwickelt:\\
	Hierfür wurde erst mal überprüft, ob die Strecke $s_1$ und $s_2$ die Länge 0 haben. Sollte dies der Fall sein, wird überprüft, ob der Startpunkt der Strecke Länge 0 auf der anderen Strecke liegt.\\
	Andernfalls wird dasselbe Verfahren genutzt, welches in Abgabe 1 verwendet wurde. Es wird mithilfe des CCW überprüft, ob sich die 2 Strecken schneiden.\\
	Sollte beide Strecken kollinear sein, wird der mögliche Überlappungspunkt zurückgegeben. Sollte die Überlappung größer als nur ein Punkt sein, wird der untere linke Punkt zurückgegeben.\\
	Wenn beide Strecken nicht kollinear sind, dann werden die beiden Strecken als Geraden behandelt. So kann man mit der Geradengleichung den Schnittpunkt der zwei Geraden berechnen werden.\\
	\[ f(x) = m*x+t \text{ und } g(x) = n*x+b \]
	\[ f(x) = g(x) \]
	Zuletzt muss noch überprüft werden ob, der Schnittpunkt der Geraden auch einer der Strecken ist.\\
	Um numerische Fehler zu reduzieren, wird bei horizontalen und vertikalen Strecken ein leicht angepasstes Verfahren genutzt. Numerische Fehler sind ein großes Problem für den Bentley-Ottmann Algorithmus.

	\section{Basis Bentley-Ottmann Algorithmus}
	
	Zu Beginn werden alle Start und Endpunkte aller Strecken $L_{InputStrecken}$ in eine EventQueue $Q_{Event}$ eingefügt. Diese \textit{EventQueue} ist eine \textit{PriorityQueue} die intern die Events auf der X-Achse und Eventtyp sortiert. \textit{START} Events haben bei gleichem X-Wert Vorrang. \textit{END} Events sind immer zuletzt dran. Die \textit{PriorityQueue} ist als ein Heap implementiert. Die Operationen besitzen deshalb auch die Komplexität \cite{b2}:
	\begin{itemize}
		\item add: $\mathcal{O}(log(n))$
		\item poll: $\mathcal{O}(1)$
	\end{itemize}
	Relevante Strecken liegen in der Sweepline $T_{Sweep}$.
	Für die Sweepline $T_{Sweep}$ wird als Datenstruktur eine Baum basierend auf einer Rot-Schwarz-Baum Implementierung verwendet. Dieser hat die Komplexität \cite{b1}:
	\begin{itemize}
		\item add: $\mathcal{O}(log(n))$
		\item remove: $\mathcal{O}(log(n))$
		\item contains: $\mathcal{O}(log(n))$
		\item get: $\mathcal{O}(log(n))$
	\end{itemize}
	Es gibt im regulären Bentley-Ottmann Algorithmus 3 Events (\textit{START}, \textit{END}, \textit{INTERSECTION}).
	Zu beginn liegen nur \textit{START} und \textit{END} Events in $Q_{Event}$.\\
	Nun wird nacheinander ein Event aus $Q_{Event}$ genommen. Dieses Event wird je nach Typ behandelt:\\
	Das Event \textit{START} fügt das neue Segment $S_{new}$ in die Sweepline $T_{Sweep}$ ein.
	Es wird überprüft ob ein Segment über oder unter $S_{new}$ liegt. Sollte dies der Fall sein wird überprüft ob diese sich mit $S_{new}$ schneiden. Bei einem gefundenen Schnittpunkten wird ein neues \textit{INTERSECTION} Event in $T_{Sweep}$ eingefügt.\\
	Bei einem \textit{END} Event endet ein Segment $S_{old}$. Es wird überprüft ob ein Segment über und unter $S_{old}$ liegt. Sollten diese existieren dann wird überprüft ob diese sich schneiden. Bei einem gefundenen Schnittpunkten wird ein neues \textit{INTERSECTION} Event in $Q_{Event}$ eingefügt. $S_{old}$ wird aus der Sweepline $T_{Sweep}$ entfernt.\\
	Bei einem \textit{INTERSECTION} Event schneiden sich die zwei Segmente s und t. Dabei werden die Positionen von s und t in $T_{Sweep}$ getauscht. Nach dem werden die Segmente r und u, die möglicherweise unmittelbar unter bzw. über t und s liegen nach Schnittpunkten untersucht. Gefundene Schnittpunkte werden der Ereigniswarteschlange hinzugefügt.

	\section{Probleme für den Algoritmus}
	Der Basisalgorithmus hat Probleme wenn folgende Voraussetzungen nicht erfüllt sind:
	\begin{enumerate}
		\item X-Koordinaten der Schnitt- und Endpunkte sind paarweise verschieden
		
		\item Segmente der Länge $> 0$
		
		\item Nur echte Schnittpunkte
		
		\item Keine Linien parallel zur y-Achse
		
		\item Keine Mehrfachschnittpunkte
		
		\item Keine überlappenden Segment
	\end{enumerate}
	Die Problemfälle werden in der Abbildung \ref{Problem} aufgezeigt.
	\begin{figure}[h]
		\begin{center}
			\includegraphics[width=6cm]{ProblemFaelle.png}
			\caption{Problemfälle für den Basis Bentley-Ottman Algorithmus}
			\label{Problem}
		\end{center}
	\end{figure}

	\section{Behandlung der Sonderfälle}
	In meiner Implementierung werden alle Sonderfälle behandelt. Manche sind einfacher zu behandeln als andere.
	\subsection{Nur echte Schnittpunkte}
	Meine Implementierung unterstützt auch unechte Schnittpunkte. Hierfür wurde vor allem die Sweepline angepasst. So wird die gewöhnliche Sweepline Implementierung, die nur aus einem Baum $T_{Sweep}$ mit Strecken besteht, mit einer Funktionalität ergänzt, die ähnlich wie bei Buckets bei einem Hashset funktioniert. So wird bei START Events überprüft, ob die Position des Startpunktes in $T_{Sweep}$ bereits existiert. Sollte dies der Fall sein, wird das Segment einfach zusätzlich in den Knoten eingefügt. Dies ist in der Abbildung \ref{figure_collision} zu sehen. Der Schlüssel der Knoten ist der aktuelle Y-Wert.\\
	\begin{figure}[h!]
		\begin{center}
			\includegraphics[width=6cm]{BaumKollision.png}
			\caption{Einfügen in die Sweepline mit Kollision}
			\label{figure_collision}
		\end{center}
	\end{figure}\\
	Leider müssen werden jedes Mal, wenn die Sweepline bewegt wird, die Elemente in $T_{Sweep}$ zu neu sortiert werden. $T_{Sweep}$ neu zu sortieren besitzt die Komplexität $\mathcal{O}(n*log(n))$. Theoretisch müsste es möglich sein, dies zu vermeiden, in dem immer nur die Nachbarn verglichen werden. Dies ist aber sehr komplex, da bei Neusortierungen die Buckets mit neu sortiert werden müssen. Ich bin überzeugt, dass es auch ohne kompletter Neusortierung gehen müsste. Der Implementierungsaufwand dafür ist aber extrem hoch. Es reicht nicht aus, immer nur die direkten Nachbarn zu betrachten, da potenziell weiter entfernte Strecken auch noch in Betracht kommen können. Deshalb muss neben den direkten Nachbarn noch weiter überprüft werden, bis man sichergehen kann, das diese sich nicht schneiden können.
	
	\subsection{X-Koordinaten der Schnitt- und Endpunkte sind paarweise identisch}
	Dieses Problem ist ein Teilproblem des vorherigen Problems und damit schon gelöst.
	So wird beim Einfügen neuer Strecken überprüft, ob bereits andere Segmente an diesem Punkt liegen. Sollte dies der Fall sein werden, wird dieser Punkt entsprechend der Anzahl der Segmente als Schnittpunkte in die Outputliste eingefügt. Außerdem immer, wenn ein Schnittpunkt gefunden wird, welcher direkt auf der Sweepline $T_{Sweep}$ liegt, wird dieser ohne ein \textit{INTERSECTION} Event zu generieren, der Outputliste hinzugefügt.
	
	\subsection{Linien parallel zur Y-Achse}
	Für Linien, die zur Y-Achse gehören, wurden in ein neues \textit{VERTICALLINE} Event erschaffen. So werden keine \textit{Start} und \textit{END} Event in die Eventqueue $Q_{Event}$ eingefügt, sondern nur das \textit{VERTICALLINE} Event.
	Vertikale Strecken schneiden sich mit allen Strecken, die aktuell die Sweepline zwischen dem Start- und Endpunkt der vertikalen Strecke schneiden. So muss nicht die gesamte Sweepline $T_{Sweep}$ untersucht werden sondern nur ein Teilbaum.
	\begin{figure}[h]
		\begin{center}
			\includegraphics[width=6cm]{Vertikal.png}
			\caption{Suche nach Schnittpunkten mit Vertikalen Strecken in einem eingeschränkten Bereich}
			\label{figure_3}
		\end{center}
	\end{figure}
	Das gilt selbstverständlich auch für anderen Vertikalen Strecken an aktueller X-Stelle. Deshalb wird eine vertikale Strecke auch an einem \textit{VERTICALLINE} Event in die Sweepline $T_{Sweep}$ eingefügt. Sie werden aber nicht in den Baum eingefügt sondern in eine separate Liste für vertikale Segmente. Sobald die Sweepline verschoben wird werden alle vertikalen Strecken aus der Liste entfernt.
		
	\subsection{Länge der Segmente gleich $0$}
	Elemente der Länge 0 werden einfach als vertikale Segmente behandelt.
	
	\subsection{Mehrfachschnittpunkte}
	Mehrfachschnittpunkte wurden behandelt. Hierfür wird die oben beschriebene Baumstruktur der Sweepline $T_{Sweep}$ genutzt. So wird an einem \textit{INTERSECTION} Event in der Sweepline gesucht, ob an der Stelle weitere Segmente durchlaufen. Sollte dies der Fall sein wird, werden direkt die korrekte Menge an Schnittpunkten der Outputliste hinzugefügt. Wichtig ist aber, dass der Fall in der Abbildung \ref{Multi} betrachtet wird. Hier muss an dem Schnittpunkt das höchste und niedrigste Segment gefunden werden und mit den benachbarten Strecken abgeglichen werden.
	\begin{figure}[h]
		\begin{center}
			\includegraphics[width=6cm]{MultiSchnitt.png}
			\caption{Problemfälle mit Multischnittpunkten}
			\label{Multi}
		\end{center}
	\end{figure}
	
	\subsection{Überlappende Segmente}
	Auch dieses Problem ist auch ein Teilproblem der nur echten Schnittpunkte. Dies kann aber noch mehr Probleme bereiten als das nur Überlappungen in einzelnen Punkten. Wichtig ist das die Funktion, die Schnittpunkte zwischen zwei Strecken findet in der Lage ist trotz Überlappung einen Schnittpunkt findet. Da eigentlich überlappende Segmente unendlich viele Schnittpunkte hat, wird einfach der untere linke Punkt gewählt.\\
	Die wichtigsten Fälle bei der Überlappung sind in Abbildung \ref{uberlap} abgebildet:
	\begin{figure}[h]
		\begin{center}
			\includegraphics[width=6cm]{ProblemUberlappen.png}
			\caption{Problemfälle mit überlappenden Elementen}
			\label{uberlap}
		\end{center}
	\end{figure}\\
	Schwierig ist vor allem der Fall 4. Hier muss am Startpunkt $p3$ der Schnittpunkt $q2/q3$ gefunden werden. Dazu müssen die Strecken $s_1$ und $s_2$, welche beide gleich weit unter Strecke $s_3$, liegen überprüft werden. Bei einer schlechten Sweepline Struktur könnte dieser Schnittpunkt nicht auffallen weil, man zufällig nur die falsche Strecke überprüft.
	
	\section{Genauigkeit}
	Ein Problem des Bentley-Ottmann Verfahrens ist es, das Schnittpunkte, die sehr nah aneinander liegen, Probleme verursachen können. Wichtig ist das die Berechnungen eine hohe Genauigkeit aufweisen. So liegen Schnittpunkte teilweise nur 4 Nachkommastellen auseinander. So sind in unserem Datensatz sehr nah aneinander, wie man in der Abbildung \ref{close} sehen kann.
	\begin{figure}[h]
		\begin{center}
			\includegraphics[width=6cm]{CloseIntersections.png}
			\caption{Sehr nah liegende Schnittpunkte}
			\label{close}
		\end{center}
	\end{figure}\\

	\section{Ergebnisse}
	Die Ergebnisse für die einzelnen Dateien sind:\\
	\vspace{0.2cm}\\
	\begin{tabular}{|c|c|}
		\hline
		Datei & Schneidende Strecken \\
		\hline
		s\_1000\_1.dat & 11 \\
		\hline
		s\_10000\_1.dat & 732 \\
		\hline
		s\_100000\_1.dat & 77126 \\
		\hline
	\end{tabular}\\
	\vspace{0.2cm}\\
	Dies stimmt auch mit den Ergebnissen aus Aufgabe 1 überein.

	\section{Komplexität}
	Die Komplexität unterscheidet sich etwas von herkömmlichen Bentley Ottman Algorithmus. Die Komplexität ist weiterhin abhängig vom Input. So gibt es normalerweise $2n + k$ Events, wobei $n$ die Anzahl der Segmente und $k$ die Anzahl der Schnittpunkten entspricht. Wobei bei Sonderfälle die Anzahl geringer Ausfallen kann. Da bei vertikalen Strecken nur ein Punkt in die \textit{Eventqueue} eingefügt wird ist die Anzahl der Events geringer. Das selbe gilt auch für Schnittpunkte die in einem Startpunkt liegen und keinen neuen Eintrag in die \textit{Eventqueue} bekommen. So gilt für $m$ nicht vertikale Strecken, $s$ Schnittpunkte ohne Überschneidungen mit anderen Events und $v$ vertikalen Strecken:
	\[ 2n + k \geq 2m + s + v \]
	Die Komplexität des Sortierens der Eventqueue beträgt $\mathcal{O}((2m+v)*log(2m+v))$ was trotzdem $\mathcal{O}(n*log(n))$ entspricht.\\
	Operationen in die \textit{Eventqueue} und der Sweepline haben die Komplexität $\mathcal{O}(log(m))$. Allerdings wird bei jedem bewegen der Sweepline die Sweepline neu sortiert. Dies resultiert leider in $\mathcal{O}(m*log(m))$.\\
	Zusammengefasst kann man den Aufwand wie folgt beschreiben:
	\[\mathcal{O}(m*log(m) + m*m*log(m)) \]
	
	\section{Performance}
	Die Laufzeiten wurden gemessen und mit anderen Methoden verglichen. Getestet wurde auf einem 6 Kern Prozessor mit Hyperthreading. Einen kleinen Nachteil besitzt der Bentley-Ottmann Algorithmus. Dieser kann nicht nur die Anzahl der Schnittpunkte berechnen, sondern er muss die Schnittpunkte selbst mindestens eine gewisse Zeit auch speichern. Um den bisherigen Code besser vergleichen zu können, berechnen die anderen Methoden jeweils die Anzahl, aber auch die Liste der Schnittpunkte.\\
	\begin{enumerate}
		\item \textbf{BF S C:} BruteForce, SingleThread, nur Anzahl Schnittpunkte
		\item \textbf{BF S L:} BruteForce, SingleThread, Liste von Schnittpunkten
		\item \textbf{BF P C:} BruteForce, MultiThread, nur Anzahl Schnittpunkte
		\item \textbf{BF P L:} BruteForce, MultiThread, Liste von Schnittpunkten
		\item \textbf{BO:} Bentley-Ottmann, Liste von Schnittpunkten
	\end{enumerate}
	\begin{table}[h!]
		\small
		\begin{tabular}{|c|c|c|c|c|c|c|}
			\hline
			Datei & BF S C & BF S L & BF P C & BF P L & BO \\
			\hline
			s\_1000\_1 & 19 & 21 & 36 & 64 & 69\\
			\hline
			s\_1000\_10 & 13 & 14 & 1 & 2 & 57\\
			\hline
			s\_10000\_1 & 1024 & 1141 & 90 & 91 & 224\\
			\hline
			s\_100000\_1 & 132712 & 145477 & 11129 & 10553 & 62049\\
			\hline
		\end{tabular}\\
	    \caption{Laufzeiten der Algorithmen}
	    \label{tabLauf}
	\end{table}
	\vspace{0.1cm}
	In der Tabelle \ref{tabLauf} man gut erkennen das der Bentley-Ottmann Algorithmus und auch der parallelisierte naive Ansatz etwas mehr Overhead besitzen und deshalb auch bei kleinen Problemen langsamer sind. Der Overhead spielt bei einer größeren Anzahl an Strecken eine geringere Rolle. Dies führt dazu das bei einer hohen Anzahl an Strecken der diese Beiden Techniken deutlich schneller werden.\\
	Für meine Implentierung des Bentley-Ottmann Algorithmus spielen weitere Faktoren als nur die Anzahl der Strecken eine Rolle.

	
	\section{Sweepline über die Zeit}
	Der durchschnittliche Füllgrad und die Anzahl der Verschiebungen der Sweepline beeinflusst die Performance maßgeblich. Die Anzahl der Schnittpunkte beeinflusst meistens auch die Anzahl der Verschiebungen der Sweepline. Dies kann man gut sehen, wenn man die Zeiten in der oberen Tabelle mit der Abbildung \ref{fullstand} vergleicht. 

	\begin{figure}[!tbp]
		\centering
		\begin{minipage}[b]{0.5\textwidth}
			\includegraphics[width=\textwidth]{s1000+1.png}
		\end{minipage}
		\hfill
		\begin{minipage}[b]{0.5\textwidth}
			\includegraphics[width=\textwidth]{s1000+10.png}
		\end{minipage}
		\hfill
		\begin{minipage}[b]{0.5\textwidth}
			\includegraphics[width=\textwidth]{s10000+1.png}
		\end{minipage}
			\hfill
		\begin{minipage}[b]{0.5\textwidth}
			\includegraphics[width=\textwidth]{s100000+1.png}
		\end{minipage}
		\caption{Füllstand der Sweepline über die Zeit}
		\label{fullstand}
	\end{figure}
	
	\begin{thebibliography}{00}
		\bibitem{b2}https://docs.oracle.com/javase/8/docs/api/java/util/PriorityQueue.html
		\bibitem{b1}https://docs.oracle.com/javase/7/docs/api/java/util/TreeMap.html
	\end{thebibliography}

\end{document}
